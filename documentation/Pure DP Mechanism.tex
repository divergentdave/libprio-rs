%% LyX 2.4.0~RC3 created this file.  For more info, see https://www.lyx.org/.
%% Do not edit unless you really know what you are doing.
\documentclass[american]{article}
\usepackage[T1]{fontenc}
\usepackage[utf8]{inputenc}
\usepackage{babel}
\usepackage{units}
\usepackage{amsmath}
\usepackage{amsthm}
\usepackage{amssymb}
\usepackage{stackrel}
\usepackage[]
 {hyperref}

\makeatletter
%%%%%%%%%%%%%%%%%%%%%%%%%%%%%% Textclass specific LaTeX commands.
\theoremstyle{plain}
\newtheorem{thm}{\protect\theoremname}
\theoremstyle{definition}
\newtheorem{defn}[thm]{\protect\definitionname}
\ifx\proof\undefined
\newenvironment{proof}[1][\protect\proofname]{\par
	\normalfont\topsep6\p@\@plus6\p@\relax
	\trivlist
	\itemindent\parindent
	\item[\hskip\labelsep\scshape #1]\ignorespaces
}{%
	\endtrivlist\@endpefalse
}
\providecommand{\proofname}{Proof}
\fi

\makeatother

\providecommand{\definitionname}{Definition}
\providecommand{\theoremname}{Theorem}

\begin{document}
\title{Differential Privacy Mechanism for \texttt{Prio3} and \texttt{PureDpDiscreteLaplace}}

\maketitle
Recall the definitions of pure differential privacy and the discrete
Laplace distribution from \cite{CKS20}.
\begin{defn}
A randomized algorithm $M:\mathcal{X}^{n}\rightarrow\mathcal{Y}$
satisfies $\varepsilon$-differential privacy if, for all $x,x^{\prime}\in\mathcal{X}^{n}$
differing on a single element and all events $E\subset\mathcal{Y}$,
we have $\mathbb{P}\left[M\left(x\right)\in E\right]\leq e^{\varepsilon}\cdot\mathbb{P}\left[M\left(x^{\prime}\right)\in E\right]$.
\end{defn}
%
\begin{defn}
The discrete Laplace distribution, with scale parameter $t$, is defined
by the following probability density function, supported on the integers.
\[
\forall x\in\mathbb{Z},\underset{X\leftarrow\mathrm{Lap}_{\mathbb{Z}}\left(t\right)}{\mathbb{P}}\left[X=x\right]=\frac{e^{\nicefrac{1}{t}}-1}{e^{\nicefrac{1}{t}}+1}\cdot e^{\nicefrac{-\left|x\right|}{t}}
\]
\end{defn}
The following differential privacy mechanism is implemented for the
combination of the \texttt{PureDpDiscreteLaplace} strategy and the
\texttt{Prio3Histogram} or \texttt{Prio3SumVec} VDAFs. Let $f\left(x\right)$
be the VDAF's aggregation function, operating over the integers. The
aggregation function produces a query result $q=f\left(x\right)\in\mathcal{Y}$.
Without loss of generality, we assume the domain $\mathcal{Y}$ is
a vector of integers, $\mathcal{Y}=\mathbb{Z}^{d}$. Let $GS_{f}$
be the global sensitivity of $f\left(x\right)$, using the replacement
definition of neighboring datasets. Let $\mathbb{F}_{p}$ be field
of prime order over which Prio3 operates. Noise is sampled from the
discrete Laplace distribution $\mathrm{Lap}_{\mathbb{Z}}\left(\nicefrac{GS_{f}}{\varepsilon}\right)$,
projected into the field, and added to each coordinate of aggregate
share field element vectors. Let $\pi_{\mathbb{F}_{p}}:\mathbb{Z}\rightarrow\mathbb{F}_{p}$
and $\pi_{\mathbb{Z}}:\mathbb{F}_{p}\rightarrow\mathbb{Z}$ be the
natural projections between the integers and field elements, where
$\pi_{\mathbb{Z}}$ maps field elements to $\left[0,p\right)$. Let
$q^{*}=f^{*}\left(x\right)\in\mathbb{F}_{p}^{d}$ be the element-wise
projections of $q$ and $f$ into the field using $\pi_{\mathbb{F}_{p}}$.
The un-noised aggregate shares produced by Prio3 are secret shares
of the query result, $q^{*}=q^{\left(0\right)}+q^{\left(1\right)}$.
Each aggregator samples noise from the discrete Laplace distribution
and adds it to the un-noised aggregate shares, and then sends the
sum as their aggregate share to the collector. If we pessimistically
assume that only one honest aggregator out of the two aggregators
is adding differential privacy noise, then the mechanism produces
$M\left(x\right)=q^{\left(0\right)}+q^{\left(1\right)}+\pi_{\mathbb{F}_{p}}\left(Z\right)=q^{*}+\pi_{\mathbb{F}_{p}}\left(Z\right)$,
where $Z_{j}\leftarrow\mathrm{Lap}_{\mathbb{Z}}\left(\nicefrac{GS_{f}}{\varepsilon}\right)$
is drawn independently for all $1\leq j\le d$.
\begin{thm}
$M\left(x\right)=\pi_{\mathbb{F}_{p}}\left(f\left(x\right)\right)+\pi_{\mathbb{F}_{p}}\left(Z\right),Z_{j}\leftarrow Lap_{\mathbb{Z}}\left(\nicefrac{GS_{f}}{\varepsilon}\right)$
satisfies $\varepsilon$-differential privacy.
\end{thm}
\begin{proof}
We will show Definition 1 holds for singleton events, where $E$ is
a set of cardinality one, then other events will follow by a union
bound.

Let $q=f\left(x\right)$, $q^{\prime}=f\left(x^{\prime}\right)$,
and $q^{*}=\vec{\pi}_{\mathbb{F}_{p}}\left(f\left(x\right)\right)$,
and let $q_{j}$, $q_{j}^{*}$, and $Z_{j}$ denote the $j$-th component
of the respective vectors. Then $M_{j}\left(x\right)=q_{j}^{*}+\pi_{\mathbb{F}_{p}}\left(Z_{j}\right)$.
Applying the probability density function of the discrete Laplace
distribution, we have:
\[
\forall j\in\left[d\right],y_{j}\in\mathbb{F}_{p},\mathbb{P}\left[M_{j}\left(x\right)=y_{j}\right]=\mathbb{P}\left[\pi_{\mathbb{F}_{p}}\left(Z_{j}\right)=y_{j}-q_{j}^{*}\right]
\]
\[
=\stackrel[k=-\infty]{\infty}{\sum}\mathbb{P}\left[Z_{j}=\pi_{\mathbb{Z}}\left(y_{j}\right)-q_{j}+kp\right]
\]
\[
=\stackrel[k=-\infty]{\infty}{\sum}\frac{e^{\nicefrac{\varepsilon}{GS_{f}}}-1}{e^{\nicefrac{\varepsilon}{GS_{f}}}+1}e^{\nicefrac{-\varepsilon\left|\pi_{\mathbb{Z}}\left(y_{j}\right)-q_{j}+kp\right|}{GS_{f}}}
\]
\[
=\frac{e^{\nicefrac{\varepsilon}{GS_{f}}}-1}{e^{\nicefrac{\varepsilon}{GS_{f}}}+1}\stackrel[k=-\infty]{\infty}{\sum}e^{\nicefrac{-\varepsilon\left|\pi_{\mathbb{Z}}\left(y_{j}\right)-q_{j}+kp\right|}{GS_{f}}}
\]

Since each $Z_{j}$ is drawn independently, the probability of the
mechanism returning some result can be found by taking the product
of the probabilities for each dimension of the result vector.
\[
\mathbb{P}\left[M\left(x\right)=y\right]=\left(\frac{e^{\nicefrac{\varepsilon}{GS_{f}}}-1}{e^{\nicefrac{\varepsilon}{GS_{f}}}+1}\right)^{d}\stackrel[j=1]{d}{\prod}\stackrel[k=-\infty]{\infty}{\sum}e^{\nicefrac{-\varepsilon\left|\pi_{\mathbb{Z}}\left(y_{j}\right)-q_{j}+kp\right|}{GS_{f}}}
\]
\[
\mathbb{P}\left[M\left(x^{\prime}\right)=y\right]=\left(\frac{e^{\nicefrac{\varepsilon}{GS_{f}}}-1}{e^{\nicefrac{\varepsilon}{GS_{f}}}+1}\right)^{d}\stackrel[j=1]{d}{\prod}\stackrel[k=-\infty]{\infty}{\sum}e^{\nicefrac{-\varepsilon\left|\pi_{\mathbb{Z}}\left(y_{j}\right)-q_{j}^{\prime}+kp\right|}{GS_{f}}}
\]

By the definition of global sensitivity, we know $\left\Vert q-q^{\prime}\right\Vert _{\ell_{1}}\le GS_{f}$.
We can break up the $\ell_{1}$ distance between $q$ and $q^{\prime}$
by dimension, and relate this sum of absolute values of differences
to the product of multiplicative factors of $e^{\left|q_{j}-q_{j}^{\prime}\right|}$,
in order to get the bound we need. Let $\delta_{j}=q_{j}-q_{j}^{\prime}$.
By the triangle inequality, $\left|\pi_{\mathbb{Z}}\left(y_{j}\right)-q_{j}+kp\right|\le\left|\pi_{\mathbb{Z}}\left(y_{j}\right)-q_{j}^{\prime}+kp\right|+\left|\delta_{j}\right|$.
Since $\varepsilon>0$ and $GS_{f}>0$, then,
\[
-\frac{\varepsilon}{GS_{f}}\left|\pi_{\mathbb{Z}}\left(y_{j}\right)-q_{j}+kp\right|\ge-\frac{\varepsilon}{GS_{f}}\left|\pi_{\mathbb{Z}}\left(y_{j}\right)-q_{j}^{\prime}+kp\right|-\frac{\varepsilon}{GS_{f}}\left|\delta_{j}\right|
\]
\[
e^{-\frac{\varepsilon}{GS_{f}}\left|\pi_{\mathbb{Z}}\left(y_{j}\right)-q_{j}+kp\right|}\ge e^{-\frac{\varepsilon}{GS_{f}}\left|\pi_{\mathbb{Z}}\left(y_{j}\right)-q_{j}^{\prime}+kp\right|-\frac{\varepsilon\left|\delta_{j}\right|}{GS_{f}}}
\]
\[
e^{\frac{\varepsilon\left|\delta_{j}\right|}{GS_{f}}}e^{-\frac{\varepsilon}{GS_{f}}\left|\pi_{\mathbb{Z}}\left(y_{j}\right)-q_{j}+kp\right|}\ge e^{-\frac{\varepsilon}{GS_{f}}\left|\pi_{\mathbb{Z}}\left(y_{j}\right)-q_{j}^{\prime}+kp\right|}
\]
\[
e^{-\frac{\varepsilon}{GS_{f}}\left|\pi_{\mathbb{Z}}\left(y_{j}\right)-q_{j}^{\prime}+kp\right|}\le e^{\frac{\varepsilon\left|\delta_{j}\right|}{GS_{f}}}e^{-\frac{\varepsilon}{GS_{f}}\left|\pi_{\mathbb{Z}}\left(y_{j}\right)-q_{j}+kp\right|}
\]

Since the above holds for a fixed $y$, $q$ and $q^{\prime}$, and
any $j$ and $k$, we can first add and then multiply inequalities
together.
\[
\stackrel[k=-\infty]{\infty}{\sum}e^{-\frac{\varepsilon}{GS_{f}}\left|\pi_{\mathbb{Z}}\left(y_{j}\right)-q_{j}^{\prime}+kp\right|}\le e^{\frac{\varepsilon\left|\delta_{j}\right|}{GS_{f}}}\stackrel[k=-\infty]{\infty}{\sum}e^{-\frac{\varepsilon}{GS_{f}}\left|\pi_{\mathbb{Z}}\left(y_{j}\right)-q_{j}+kp\right|}
\]
\[
\stackrel[j=1]{d}{\prod}\stackrel[k=-\infty]{\infty}{\sum}e^{-\frac{\varepsilon}{GS_{f}}\left|\pi_{\mathbb{Z}}\left(y_{j}\right)-q_{j}^{\prime}+kp\right|}\le\stackrel[j=1]{d}{\prod}e^{\frac{\varepsilon\left|\delta_{j}\right|}{GS_{f}}}\stackrel[k=-\infty]{\infty}{\sum}e^{-\frac{\varepsilon}{GS_{f}}\left|\pi_{\mathbb{Z}}\left(y_{j}\right)-q_{j}+kp\right|}
\]
\[
\stackrel[j=1]{d}{\prod}\stackrel[k=-\infty]{\infty}{\sum}e^{-\frac{\varepsilon}{GS_{f}}\left|\pi_{\mathbb{Z}}\left(y_{j}\right)-q_{j}^{\prime}+kp\right|}\le e^{\frac{\varepsilon\stackrel[j=1]{d}{\sum}\left|\delta_{j}\right|}{GS_{f}}}\stackrel[j=1]{d}{\prod}\stackrel[k=-\infty]{\infty}{\sum}e^{-\frac{\varepsilon}{GS_{f}}\left|\pi_{\mathbb{Z}}\left(y_{j}\right)-q_{j}+kp\right|}
\]
\[
\stackrel[j=1]{d}{\prod}\stackrel[k=-\infty]{\infty}{\sum}e^{-\frac{\varepsilon}{GS_{f}}\left|\pi_{\mathbb{Z}}\left(y_{j}\right)-q_{j}^{\prime}+kp\right|}\le e^{\frac{\varepsilon}{GS_{f}}\left\Vert q-q^{\prime}\right\Vert _{\ell_{1}}}\stackrel[j=1]{d}{\prod}\stackrel[k=-\infty]{\infty}{\sum}e^{-\frac{\varepsilon}{GS_{f}}\left|\pi_{\mathbb{Z}}\left(y_{j}\right)-q_{j}+kp\right|}
\]
Then, since $\left\Vert q-q^{\prime}\right\Vert _{\ell_{1}}\le GS_{f}$,
\[
\stackrel[j=1]{d}{\prod}\stackrel[k=-\infty]{\infty}{\sum}e^{-\frac{\varepsilon}{GS_{f}}\left|\pi_{\mathbb{Z}}\left(y_{j}\right)-q_{j}^{\prime}+kp\right|}\le e^{\frac{\varepsilon}{GS_{f}}GS_{f}}\stackrel[j=1]{d}{\prod}\stackrel[k=-\infty]{\infty}{\sum}e^{-\frac{\varepsilon}{GS_{f}}\left|\pi_{\mathbb{Z}}\left(y_{j}\right)-q_{j}+kp\right|}
\]
\[
\stackrel[j=1]{d}{\prod}\stackrel[k=-\infty]{\infty}{\sum}e^{-\frac{\varepsilon}{GS_{f}}\left|\pi_{\mathbb{Z}}\left(y_{j}\right)-q_{j}^{\prime}+kp\right|}\le e^{\varepsilon}\stackrel[j=1]{d}{\prod}\stackrel[k=-\infty]{\infty}{\sum}e^{-\frac{\varepsilon}{GS_{f}}\left|\pi_{\mathbb{Z}}\left(y_{j}\right)-q_{j}+kp\right|}
\]
This shows that $\mathbb{P}\left[M\left(x^{\prime}\right)=y\right]\le e^{\varepsilon}\cdot\mathbb{P}\left[M\left(x\right)=y\right]$.
By applying union bounds, then $\mathbb{P}\left[M\left(x^{\prime}\right)\in E\right]\le e^{\varepsilon}\cdot\mathbb{P}\left[M\left(x\right)\in E\right]$
as well, and thus $M\left(x\right)$ satisfies $\varepsilon$-differential
privacy.
\end{proof}
\begin{thebibliography}{1}
\bibitem{CKS20}Canonne, C. L., Kamath, G., and Steinke, T., ``The
Discrete Gaussian for Differential Privacy'', 2020, <\href{https://arxiv.org/abs/2004.00010}{https://arxiv.org/abs/2004.00010}>.

\end{thebibliography}

\end{document}
